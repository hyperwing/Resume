% LaTeX file for resume 
% This file uses the resume document class (res.cls)
\documentclass[]{res} 
%\renewcommand{\sectionfont}{\small}
\usepackage{multicol}


%\usepackage{helvetica} % uses helvetica postscript font (download helvetica.sty)
%\usepackage{newcent}   % uses new century schoolbook postscript font 
\newsectionwidth{0pt}  % So the text is not indented under section headings
\setlength{\textheight}{10in} % set text height big enough for box
\topmargin=-.8in       % to start box .5in from top of page
\oddsidemargin=-.8in   % to start box .5in from left of page

\setlength{\multicolsep}{3.0pt plus 2.0pt minus 2pt}% 50% of original values




\begin{document}
	
	%%%%%%%%%%%%%%%%%%%%%%%%%%%%%%%%%%%%%%%%%%%%%%%%%%%%%%%%%%%%%%%%%%%%%%%%%%%%
	% The following lines define \boxaround, used to draw a box on the page.
	% The parameter is the entire text of the resume. Must fit on one page!
	%
	% \boxaroundhmargin is the left & right margin around the text inside the box.
	% \boxaroundvmargin is the top & bottom margin around the text inside the box.
	% \boxrulethickness controls thickness of line used to draw the box.
	% You can change these 3 things in the lines below:
	%%%%%%%%%%%%%%%%%%%%%%%%%%%%%%%%%%%%%%%%%%%%%%%%%%%%%%%%%%%%%%%%%%%%%%%%%%%%%
	\newdimen\boxrulethickness\newdimen\boxaroundhmargin\newdimen\boxaroundvmargin
	\boxrulethickness=.5pt        %controls thickness of line 
	\boxaroundhmargin=.5in        % about a half inch
	\boxaroundvmargin=5pt        % to fit more text on page, make this smaller
	%%%%%%%%%%%%%%%%%%%%%%%%% Don't read this stuff %%%%%%%%%%%%%%%%%%%%%%%%%%%%%%
	\hsize=7.8in \vsize=10.5in             % use bigger dimensions for box
	\newbox\MACboxA  \newdimen\MACdimenA
	% \borderandboxit is used inside \boxaround:
	\def\borderandboxit#1#2#3{\vbox{\hrule height#2\hbox{\vrule width#2\hskip#1\hskip-#2%
				\vbox{\vskip#1\relax#3\vskip#1}\hskip#1\hskip-#2\vrule width#2}\hrule height#2}}
	%
	\long\def\boxaround#1{\vskip6pt
		{\MACdimenA=\hsize \advance\MACdimenA by-\boxaroundhmargin
			\advance\MACdimenA by-\boxaroundhmargin   % once for each side
			\setbox\MACboxA=\hbox to \hsize{\hskip\boxaroundhmargin%\hss
				\vbox{\hsize=\MACdimenA
					\vskip\boxaroundvmargin #1
					\vskip\boxaroundvmargin}\hss}%
			\borderandboxit{0pt}\boxrulethickness{\box\MACboxA}}%
		\vskip2pt plus0pt minus0pt
	}
	%%%%%%%%%%%%%%%%%%%  End of \boxaround macro %%%%%%%%%%%%%%%%%%%%%%%%%%%%%%%%%
	
	\boxaround{ 
	
		\begin{multicols}{4}
			\textbf{David A. Wing}\\ \ \ \  wing.49@osu.edu\\ \ \ \ \ \ (614)327-6763\\github.com/hyperwing
			\noindent
		\end{multicols}
					
			\section{\sl  \textbf{Education}}			
			The Ohio State University\\ 
			B.S.,  Computer Science and Engineering, Honors and Scholars
			\hfill expected July 2020 

			\section{\sl \textbf{Honors}}	
			%Dean's List for Engineering Students  \hfill 2016-2017 \\
			Morril Excellence Scholarship   \hfill  2015-2020 \\
			Engineering Dean's Scholarship  \hfill  2015 
			
			\section{\sl  \textbf{Qualifications}}
			\begin{multicols}{3}
			Java\\
			Python\\
			JavaScript \\
			C\#\\
			C++

			
			SQL Server\\ 
			AWS(S3, Lambda, Glue)\\
			Android Development Studio\\ 
			Node.js\\  Mocha\\ RHEL \\Raspian
			\\AngularJS\\Powershell\\VB.NET
			\end{multicols}
						
			\section{\sl\textbf{Work Experience}}
			
			\begin{ncolumn}{2}	
			{\it Capital One}, McLean, VA  & \hfill   5/19-8/19
			\end{ncolumn}\\
			Technology Internship Program - ATM Division \\
			%{\textit{\footnotesize{Technologies Used: Python, C++, Wireshark, XFS-CEN API}}}\\
			Reverse Engineered USB Drivers for ATMs and wrote custom USB drivers. Created a script to run API calls from XFS-CEN API. Collected USB packets using the script from the ATM with Wireshark. Reverse engineered the USB signals to be able to run arbitrary code on the ATM and create custom drivers in Python. Wrote an automation testing framework for testing of all ATM hardware.
			
			
			\begin{ncolumn}{2}
			{\it Asurion}, Nashville, TN  & \hfill   5/18-7/18
			\end{ncolumn}\\
			Software Engineering Intern - Sub-billing Division\\
			%{\textit{\footnotesize{Technologies Used: Node.js, Python, Mocha, AWS, JavaScript}}}\\
			Worked with back office transactions, streamlining account processing by creating internal tools in Node.js and Python. Created transaction records and histories for accounts by combining and inferring existing data. Wrote a unit test suite in Mocha for all of my tools and resolved critical issues in existing AWS applications. Led a team presenting a technical business case on new LOBs to executives at Asurion.
			
			\begin{ncolumn}{2}
			{\it Innovative Systems}, Pittsburgh, PA  & \hfill   5/17-8/17
			\end{ncolumn}\\
			Software Engineering Intern - R\&D Division \\
			%{\textit{\footnotesize{Technologies Used: Java, VB.Net, SQL Server, Oracle}}}\\
			Created tools used in the daily deployment of database updates and migrations. Converted several stored procedures into a more extensible format. Used industry software design practices and documentation. Envisioned and developed a cohesive product that is currently in production use. Using VB.NET and Java, the product took user instructions and ran through a complex procedure to transfer data used in daily deployment.
			
			
			\begin{ncolumn}{2} 
			{\it The Ohio State University}, Columbus, OH & \hfill  4/16-11/16 
			\end{ncolumn}\\
			Student Information Technology Worker- Chemistry Department \\
			%{\textit{\footnotesize{Technologies Used: Powershell, Raspian, RedHat, Windows}}}\\
			Created scripts in Powershell and Python deployed for use for software automation across the department. System Admin for RHEL and Raspian for hardware and software deployment.
			

			\section{\sl  \textbf{Software Projects}}
			\begin{ncolumn}{2} 
			\textit{Inter-collegiate Programming Competition - ACM} & \hfill  2016-2019 
			\end{ncolumn}\\
			Competed in an algorithmic programming competition for college students. Represented Ohio State in the East Central Conference Event and placed in top half of participants.
			
			\begin{ncolumn}{2} 
			\textit{OHI/O Hackathon} & \hfill  2015-2019 
			\end{ncolumn}\\
			Created an Angular application that acts as a driving companion app for cars that tracks fuel efficiency and other metrics pulling from internal device components for a Honda challenge. Application successfully built for cross platform usage, and derives statistics from kernel level calls.
			
			\textit{FIRST Robotics Team - Programming Division}\\
%			Worked as the lead programmer on the varsity robotics team. Used LabVIEW to create code for a robot that utilized a variety of mechanisms, such as pneumatics, CIMs, and real time field tracking using the Kinect.
			Worked as the lead programmer on the varsity robotics team. The team advanced to the highest stage of the competition, and performed in the top half of the participants.
			
			\textit{Unity3D Game Development}\\
			%{\textit{\footnotesize{Technologies Used: C\texttt{\#}, TypeScript}}}\\
			Developed several Unity3D games for Android, Windows and Web Browsers. One game was a platform side scroller, featuring unlimited tile generation and score keeping along with various powerups, each influencing the run of play. Ended with limited release on the title.
			
			\textit{Android Development}\\
			%{\textit{\footnotesize{Technologies Used: Android Studios, Java}}}\\
			Developed several native Android applications as proof of concept ideas, or as learning tools. Applications created for hackathons and personal usage to solve common inconveniences.
			

			



	}
		
%	\boxaround{ 
%		\section{\sl  \textbf{Side Projects}}
%
%
%			
%		\textit{Raspberry Pi Automation}\\
%		{\textit{\footnotesize{Technologies Used: Python, Raspian, Java, Apache Server, Telegram}}}\\
%			Developed Raspberry Pi project for automating and controlling LED lights from a mobile device. The application invoked Python calls from a Java frontend, with an Apache server backend running the service.\\
%			Developed a Telegram messaging application. Using the Telegram API and python, the application would be triggered from a physical button press and send pictures to groups randomized from images sent to the bot. 
%			
%		\textit{Project Euler \& Kattis}\\
%			Practice algorithmic problem solving on sites such as Project Euler and Kattis. Programs are written for the best performance and space utilization on frequently completed problems. 
%			
%		\textit{Alexa Skills}\\
%		{\textit{\footnotesize{Technologies Used: TypeScript, Alexa, AWS}}}\\
%			Developed an Alexa skill to read, create, and automatically purchase ingredients for recipes at VolHacks 2018.
%			
%
%		\textit{Audio Visual Technician}\\
%			Worked as an Audio Visual technician, in charge of maintaining, running and training other students to work on large events. 
%			
%		}
%	

		 %    end the material being boxed.
\end{document}


